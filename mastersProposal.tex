\documentclass[10pt,letterpaper,oneside]{article}

\input{./latexGoodPractices/preamble}

%==============================================================
% FILL THIS SECTION

\newcommand{\projectTitle}{Use a Descriptive Title}
\newcommand{\projectStudent}{Your Name}
\newcommand{\projectStudentNIP}{xxx xxx xxx}
\newcommand{\projectDate}{\today} % or manually: November 23, 2017

\newcommand{\projectSupervisor}{Prof. F. Pomerleau}
\newcommand{\projectCoSupervisor}{Prof. P. Gigu\`{e}re}

% Change to your specific bibliography file
\addbibresource{./latexGoodPractices/exampleReferences.bib}
%\addbibresource{./references.bib}
%==============================================================




% ---------------------------------------------------------------
% Load style
%----------------------------------------
% Page style

% Set the page size
\addtolength{\hoffset}{-1.0in} \addtolength{\voffset}{-0.75in}
\setlength{\textwidth}{7in} \setlength{\textheight}{8.25in}
\setlength{\headheight}{0.6in}
\setlength{\headsep}{0.2in}

\setlength{\footskip}{40pt}
\setlength{\fboxsep}{12pt}

% Set the paragraph skip
\setlength{\parskip}{3pt}

% Access to a counter for the number of pages
\usepackage{lastpage}

% To allow text justify on the right
\usepackage{ragged2e}

%----------------------------------------
% Title style

\newcommand{\makeCustomTitle}
{
\begin{center}
\LARGE{\textbf{\projectTitle{}}}
\\
\vspace{5pt}
\normalsize{\projectStudent{} (\projectStudentNIP{})}
\\
\projectDate{}
\end{center}
\begin{flushright}
\footnotesize{Supervised by \projectSupervisor{} 
\\
and co-supervised by \projectCoSupervisor{}}
\end{flushright}
}

%----------------------------------------
% Section style
\usepackage{sectsty}

% Set the section labeling font
\allsectionsfont{\textsf\bfseries}

%----------------------------------------
% Caption style
\usepackage[font=small, labelfont=bf, skip=5pt]{caption}

%----------------------------------------
% header style
\usepackage{fancyhdr}



% Define the title page style
\fancypagestyle{titlePage}{%
\fancyhf{}%

\fancyhead[L]{\includegraphics[height=0.45in]{UL_N}}
\fancyhead[C]{\raisebox{0.2in}{\textsc{Master's Project Proposal}}}
\fancyhead[R]{\includegraphics[height=0.45in]{norlab_logo_acronym_dark}}
\fancyfoot[C]{\thepage/\pageref*{LastPage}}

\renewcommand{\headrulewidth}{0.1pt}
\renewcommand{\footrulewidth}{0.2pt}
}

% Define the page style for the other pages
\fancypagestyle{plain}{%
\fancyhf{}
\fancyhead[L]{\projectTitle{}}
\fancyfoot[C]{\thepage/\pageref*{LastPage}}
\renewcommand{\headrulewidth}{0.1pt}
\renewcommand{\footrulewidth}{0.1pt}
}

% Set the page style for all the document except the first page
\pagestyle{plain}

%----------------------------------------
% footnote style
\usepackage{fnpos}
% Fix the footnotes location
\makeFNbottom \makeFNbelow

% ---------------------------------------------------------------
% Author

\author{\projectStudent{} \\
       Laval University\\
       1065, av. de la Médecine \\
       Quebec, Qc \\
       Canada G1V 0A6 \\
}

% ---------------------------------------------------------------
% PDF setup
\hypersetup{%
    pdftitle={\projectTitle},
    pdfauthor={\@author},
    pdfkeywords={research, project, robotics, norlab, Northern Robotics Lab, Master's},
    pdfsubject={},
    pdfstartview={},
    urlcolor=cyan,
    linkcolor=red,
}%

% ---------------------------------------------------------------
% produce Gantt Chart
\usepackage{pgfgantt}

% ---------------------------------------------------------------


% Customize enumerate list
\usepackage{enumerate}



%================================================================
\begin{document}
\makeCustomTitle
\thispagestyle{titlePage}

% ---------------------------------------------------------------
\section{Introduction}

Write one sentence per line.
This helps to spot too long sentences and simplify diff on files.
\lightlipsum[1-2]

% ---------------------------------------------------------------
\section{Research Problem}

Whenever you can, avoid to write at the first person (e.g., I, my).

\lightlipsum[1]

\begin{center}
\emph{
Don't hesitate to highlight your research question like this.
}
\end{center}

\lightlipsum[1]

% ---------------------------------------------------------------
\section{Related Work}
Aim for 10 to 15 core articles (i.e., related to the theory you want build on) and 3 to 5 satellite articles (i.e., big picture or motivation for the problematic).
Here some examples of references \cite{Pomerleau2013,Pomerleau2014}.

\lightlipsum[1-3]

If you do use a figure, always link your figure in the text (e.g., \autoref{fig:overview}).
Do not repeat the caption as a shortcut.

\begin{figure}[htb]
\centering
\includegraphics[draft, width=0.3\textwidth]{./figs/overview.pdf}
\caption{
Replace the file \texttt{./figs/overview.pdf} with a photo or diagram.
Use a descriptive caption so the reader understand what is going on without searching in the text.
This is just an example, you don't need to have figures for your Master's project proposal.
Never use figures from another article that you are not the author.
Before investing time in a figure, consult your supervisor.
}
\label{fig:overview}
\end{figure}

\lightlipsum[1-2]

% ---------------------------------------------------------------
\section{Objectives and Methodology}
\label{sec:objectives}

\emph{Objective one}:

\begin{enumerate}[{1.}1]
\item First step
\item Second step
\item ...
\end{enumerate}

\emph{Objective two}:

\begin{enumerate}[{2.}1]
\item First step
\item Second step
\item ...
\end{enumerate}

% ---------------------------------------------------------------
\section{Schedule}

Always link your table in the text (e.g., \autoref{fig:gantt}).

\lightlipsum[1]


\begin{figure}[htbp]
  \centering
  
%---------------------------------------------------------------
% CONFIGURATION

\definecolor{navyblue}{RGB}{21,80,130}
\setganttlinklabel{f-s}{}

\begin{ganttchart}[
     %Specs
     y unit title=0.5cm,
     y unit chart=0.5cm,
     x unit=1.3cm,
     vgrid={*{2}{draw=none},*{1}{black}},
     title height=1,
     title label font=\bfseries\footnotesize,
     % bar
     bar/.style={fill=navyblue},
     bar height=0.7,
     bar label font=\footnotesize,
     bar label node/.append style={left=10pt},
     % group
     group right shift=0,
     group top shift=0.7,
     group height=.3,
     group peaks width={0.2},
     group peaks height={0.3},
     group label node/.append style={left=10pt},
     group label font=\bfseries\footnotesize,
     % milestone
     milestone/.append style={xscale=0.4, yscale=1},
     milestone label node/.append style={left=10pt},
     milestone label font=\itshape\footnotesize
     ]{1}{6}
    \gantttitle[]{20XX}{3}
    \gantttitle[]{20XX}{3} 
    \\              
    \gantttitle{W}{1} \gantttitle{S}{1} \gantttitle{A}{1}
    \gantttitle{W}{1} \gantttitle{S}{1} \gantttitle{A}{1}
    \\
    %--------------------------------------       
    \ganttgroup{Objective 1: Stuff}{1}{3}\\ 
    
    \ganttbar{Task 1}{1}{2}
    \ganttbar[inline, bar label font/.append=\color{white}]{Extra info}{1}{2}\\
    \ganttmilestone{Milestone (M1)}{1.5}\\
	
	\ganttbar{Task 2}{2}{3}
	\ganttbar[inline, bar label font/.append=\color{white}]{more info}{2}{3}\\
	\ganttmilestone{Milestone (M2)}{3}\\
	
    %--------------------------------------       
    \ganttgroup{Objective 2: More stuff}{3}{6} \\ 
    
    \ganttbar{Task 3}{3}{4}
    \ganttbar[inline, bar label font/.append=\color{white}]{stuff}{3}{4}\\
    
    \ganttmilestone{Milestone (M3)}{4}\\
    
    \ganttbar{Task 4}{4}{6}
    \ganttbar[inline, bar label font/.append=\color{white}]{tmp}{4}{6}\\
    \ganttmilestone{Milestone (M4)}{6}\\
    
    %--------------------------------------       
    \ganttgroup{Objective 3: Field Testing}{1}{6} \\
    \ganttbar[name=i1]{Integration}{2}{2}
    \ganttbar[name=i2]{}{5}{5}
    \\
    \ganttbar[name=t1]{Deployment and tests on campus}{3}{3}
    \ganttbar[name=t2]{}{6}{6}
    \\
    \ganttbar[name=e1]{Deployment and experiments in forests}{4}{4}
    \\
   % links
    \ganttlink[link type=f-s]{i1}{t1}
    \ganttlink[link type=f-s]{t1}{e1}
    \ganttlink{e1}{i2}
    
    \ganttlink[link type=f-s]{i2}{t2}

\end{ganttchart}

    \caption{%
  Gantt chart containing the objectives to be accomplished during this project.
  The numbers refer to the objectives listed in \autoref{sec:objectives}.
  W= winter semester, S = summer semester, and A: Autumn semester.
  \textbf{Note:} Change the file \texttt{./gantt.tex} to edit this table.
  }
  \label{}
\end{figure}

% ---------------------------------------------------------------
\section{Conclusion}

\lightlipsum[1]

% ---------------------------------------------------------------
\printbibliography


\end{document}
